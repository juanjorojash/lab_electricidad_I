\documentclass[12pt]{article}
\usepackage[spanish,activeacute]{babel}
\usepackage[margin=1in]{geometry}
\usepackage{amsmath,amssymb}
\usepackage{multicol}
\usepackage{siunitx}
\usepackage[american,RPvoltages]{circuitikz} %Este se usa para hacer circuitos
\usepackage{float}
\usepackage{hyperref}
\usepackage{tabularx}
\usepackage{booktabs}
\newcolumntype{C}{>{\centering\arraybackslash}X}


% *** GRAPHICS RELATED PACKAGES ***
%
\usepackage{graphicx}
\graphicspath{{../pdf/}{../png/}}
\DeclareGraphicsExtensions{.pdf,.jpg,.png}
\usepackage{subfigure}


\begin{document}
\renewcommand{\tablename}{Tabla}
\renewcommand{\listtablename}{Índice de tablas}
\noindent
\begin{tabularx}{\linewidth}{Xr}
\textbf{Lab. de Electricidad I}& \textbf{Mediciones del Laboratorio virtual} 
\end{tabularx}\\

\noindent\rule[2ex]{\textwidth}{2pt}
\centering
\textbf{Lab \# 5: Teorema de Superposición}\\[12pt]
\noindent\rule[2ex]{\textwidth}{2pt}

\begin{table}[H]
	\caption{Valores medidos de las resistencias utilizadas}
	\centering
	\vspace{0.5cm}
	\begin{tabularx}{10cm}{lCC}
	    \toprule
		Teórico (\si{\kilo\ohm}) & Real (\si{\kilo\ohm}) & Comentarios\\
		\midrule
		1 & 1 &\\
		10 & 10,11 &\\
		4,7 & 4,68 & horizontal (a-b)\\
		4,7 & 4,67 & vertical\\
		2,2 & 2,18 &\\
		\bottomrule
	\end{tabularx}
\end{table}

\begin{table}[H]
	\caption{Valores de $V_{ab}$ medidos con una fuente a la vez}
	\centering
	\vspace{0.5cm}
	\begin{tabularx}{6cm}{lC}
	    \toprule
		Solo la de \SI{10}{\volt} & \SI{6.03}{\volt}\\
		Solo la de \SI{20}{\volt} & \SI{-9.08}{\volt}\\
		$\sum$ de fuentes & \SI{-3.05}{\volt}\\
		\midrule
		Ambas fuentes & \SI{-2.99}{\volt}\\
		\bottomrule
	\end{tabularx}
\end{table}

\begin{table}[H]
	\caption{Valores de $I_{ab}$ medidos con una fuente a la vez}
	\centering
	\vspace{0.5cm}
	\begin{tabularx}{6cm}{lC}
	    \toprule
		Solo la de \SI{10}{\volt} & \SI{1.29}{\ampere}\\
		Solo la de \SI{20}{\volt} & \SI{-1.93}{\ampere}\\
		$\sum$ de fuentes & \SI{-0.64}{\volt}\\
		\midrule
		Ambas fuentes & \SI{-0.64}{\volt}\\
		\bottomrule
	\end{tabularx}
\end{table}

\end{document}