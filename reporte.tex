%%%%%%%%%%%%%%%%%%%%%%%%%%%%%%%%%%%%%%%%%
% Tecnologico de Costa Rica/Instructivo de Laboratorio de Electricidad I
% LaTeX Template
% Version 3.1 (25/3/14)
%
% This template has been downloaded from:
% http://www.LaTeXTemplates.com
%
% Original author:
% Linux and Unix Users Group at Virginia Tech Wiki 
% (https://vtluug.org/wiki/Example_LaTeX_chem_lab_report)
%
% License:
% CC BY-NC-SA 3.0 (http://creativecommons.org/licenses/by-nc-sa/3.0/)
%
%%%%%%%%%%%%%%%%%%%%%%%%%%%%%%%%%%%%%%%%%

%----------------------------------------------------------------------------------------
%	PACKAGES AND DOCUMENT CONFIGURATIONS
%----------------------------------------------------------------------------------------

\documentclass[12pt,letterpaper]{article}
\usepackage{url}
\usepackage{float}
\usepackage[utf8]{inputenc}
\usepackage[spanish,activeacute]{babel} %Permite la escritura intiutiva en español
\usepackage[version=3]{mhchem} % Package for chemical equation typesetting
\usepackage{siunitx} % Provides the \SI{}{} and \si{} command for typesetting SI units
\usepackage{graphicx} % Required for the inclusion of images
\usepackage[siunitx,RPvoltages,american]{circuitikz} %Este se usa para hacer circuitos
%\usepackage{natbib} % Required to change bibliography style to APA
\usepackage{amsmath} % Required for some math elements 
\usepackage{array}
\usepackage{tabularx}
\newcolumntype{C}{>{\centering\arraybackslash}X}
\usepackage{booktabs}
\usepackage{geometry}                         
\geometry{left=18mm,right=18mm,top=21mm,bottom=21mm,headheight=15pt} % Tamaño del área de escritura de la página
\usepackage{lastpage}
\usepackage{hyperref}
\hypersetup{
    colorlinks=true,
    linkcolor=blue,
    filecolor=magenta,      
    urlcolor=blue,
    citecolor=blue,
    pdfpagemode=FullScreen,
}
\setlength\parindent{0pt} % Removes all indentation from paragraphs

\renewcommand{\labelenumi}{\alph{enumi}.} % Make numbering in the enumerate environment by letter rather than number (e.g. section 6)
\usepackage{fancyhdr}
\pagestyle{fancy}
\lhead{Laboratorio de Electricidad I}
\rhead{Laboratorio 1}
\lfoot{Escuela de Ingeniería Electromecánica}
\cfoot{\thepage\ de \pageref{LastPage}}

\rfoot{\begin{picture}(0,0) \put(-60,0){\includegraphics[width=25mm]{logo}} \end{picture}}


%\usepackage{times} % Uncomment to use the Times New Roman font
\newcommand{\obj}{Objetivos}
\newcommand{\inv}{Investigación previa}
\newcommand{\mat}{Materiales y equipo}
\newcommand{\pro}{Procedimiento}
\newcommand{\capacidad}{Al finalizar este laboratorio el estudiante estará en capacidad de:}
\newcommand{\antesde}{Antes de empezar el laboratorio presente el siguiente cuestionario lleno.}
%----------------------------------------------------------------------------------------
%	DOCUMENT INFORMATION
%----------------------------------------------------------------------------------------

\addto\captionsspanish{\renewcommand{\chaptername}{Laboratorio}}
\addto\captionsspanish{\renewcommand{\tablename}{Tabla}}
\begin{document}


\begin{titlepage}

\begin{center}
\vspace*{-0.5in}
\begin{figure}[htb]
\begin{center}
\includegraphics[width=11cm]{logo}
\end{center}
\end{figure}
\vspace*{0.4in}
\begin{Large}
ESCUELA DE INGENIERIA ELECTROMECANICA\\
\vspace*{0.15in}
AREA ELECTRICA\\
\vspace*{0.8in}
\end{Large}
\vspace*{0.2in}
\begin{Large}
\textbf{Reporte: Laboratorio I} \\
\end{Large}
\textbf{Empleo y lectura de instrumentos de medición eléctrica para corriente y voltaje}
\vspace*{0.3in}\\
\begin{center}
    \begin{tabular}{c|c}
        Nombre Apellido & Carné  \\
        Nombre Apellido & Carné
    \end{tabular}
\end{center}
\vspace*{2.5in}
\begin{Large}
\textbf{\today}\\
\end{Large}
\rule{80mm}{0.1mm}\\
\vspace*{0.1in}
\end{center}

\end{titlepage}

\tableofcontents
\newpage

\section{\obj}
\begin{itemize}
\item Comprender las limitaciones de lectura en los voltímetros y amperímetros de corriente continua en varias escalas.
\item Calcular la resistencia equivalente de un elemento utilizando el método indirecto de la Ley de Ohm.
\item	Comprobar experimentalmente la Ley de Ohm.
\item	Demostrar el concepto de linealidad en una resistencia.
\end{itemize}

\section{\inv}
\antesde
\begin{enumerate}
\item	¿Cómo se debe conectar un multímetro para medir la corriente en una carga? 
\item	¿Cómo se debe conectar un multímetro para medir el voltaje en una carga?
\item	Según las especificaciones de su multímetro y para los distintos rangos de medición, defina cuál es el porcentaje de error en las lecturas de voltaje y corriente.
\item	¿Qué significa la linealidad de un dispositivo eléctrico? ¿Qué consecuencias tiene esta?
\end{enumerate}

\section{\mat}
\textbf{A suministrar por la Escuela:}
\begin{itemize}
\item 1 multímetro digital
\item 1 socket para bombillo incandescente
\item 1 bombillo incandescente
\item 1 resistencia de potencia de 100 \si{\ohm} en 12 \si{\watt}
\item 10 cables conectores medianos
\item 1 fuente variable de voltaje DC
\end{itemize}
\textbf{A suministrar por el estudiante:}
\begin{itemize}
\item 1 multímetro digital
\item 1 protoboard
\item Cable de interconexión macho-macho
\item 3 resistencias de carbón con valores entre \SI{1}{\kilo\ohm} y \SI{10}{\kilo\ohm}
\end{itemize}
\section{\pro}
\begin{enumerate}
\item Realice las conexiones del circuito tal y como se indica en la Figura \ref{fig:L1F1}.
\begin{figure}[H]
\centering
\begin{circuitikz} 
\draw
(0,0) 	
    to[V, l=$V_f$] 
(0,3)
	to[ammeter] 
(3,3)
	to[lamp, l_=B, *-*] 
(3,0) -- (0,0)
(3,3) -- (5,3)
    to[voltmeter] 
(5,0) -- (3,0)
		
;
\end{circuitikz}
\caption{Medición de corriente y voltaje en un circuito.}
\label{fig:L1F1}
\end{figure}

\item Encienda la fuente con un voltaje de \SI{32}{\volt} y espere unos minutos mientras el bombillo se calienta. 
\item Una vez que el bombillo calentó mida la corriente que circula por el bombillo y el voltaje que cae en este, anote el resultado en la Tabla \ref{tab:L1T1}.
\item	Repita las mediciones utilizando los valores de voltaje que indica la Tabla \ref{fig:L1F1} y complete la misma. Calcule indirectamente para cada caso el valor de la resistencia interna del bombillo.
\item	Grafique el comportamiento de la resistencia interna del bombillo y analice el comportamiento. 
\item	Cambie ahora el bombillo por una resistencia de \SI{100}{\ohm} (\SI{12}{\watt}), repita los pasos anteriores empezando con cero voltios en la fuente, llene la Tabla \ref{tab:L1T2}.
\item  Grafique el comportamiento de la resistencia y analice el comportamiento. 
\item Realice las conexiones del circuito tal y como se indica en la Figura \ref{fig:L1F2}.
\item Encienda la fuente con un voltaje de \SI{30}{\volt}.
\item Realice las mediciones de $I$, $V_{R1}$, $V_{R2}$, $V_{R3}$.
\item Averigüe los valores teóricos de las resistencias usando el código de colores. 
\item Llene la Tabla \ref{tab:L1T3}
\item Realice las conexiones del circuito tal y como se indica en la Figura \ref{fig:L1F3}.
\item Encienda la fuente con un voltaje de \SI{10}{\volt}.
\item Realice las mediciones de $V$, $I_{R1}$, $I_{R2}$, $I_{R3}$.
\item Averigüe los valores teóricos de las resistencias usando el código de colores. 
\item Llene la Tabla \ref{tab:L1T4}
\end{enumerate}

\begin{figure}[H]
\centering
\begin{circuitikz} 
\draw
(0,0) 	
    to[V, l=\SI{30}{\volt}, i=$I$] 
(0,3)
	to[R, l=$R_1$, v=$V_{R1}$] 
(3,3)
	to[R, l=$R_2$, v=$V_{R2}$] 
(6,3) 
    to[R, l=$R_3$, v=$V_{R3}$] 
(9,3) -- (9,0) -- (0,0)
;
\end{circuitikz}
\caption{Circuito en serie con tres resistencias}
\label{fig:L1F2}
\end{figure}

\begin{figure}[H]
\centering
\begin{circuitikz} 
\draw
(0,0) 	
    to[V, l=\SI{10}{\volt}] 
(0,3)
	to[short] 
(2,3)
	to[R, l=$R_1$, i=$I_{R1}$] 
(2,0)
    --
(0,0)
(2,3)
    --
(4,3)
    to[R, l=$R_2$, i=$I_{R2}$] 
(4,0)
    --
(2,0)
(4,3)
    --
(6,3)
    to[R, l=$R_3$, i=$I_{R3}$]
(6,0)
    --
(4,0)
;
\end{circuitikz}
\caption{Circuito en paralelo con tres resistencias}
\label{fig:L1F3}
\end{figure}

\section{Resultados}

\begin{table}[H]
	\caption{Valores experimentales de corriente y voltaje del bombillo}
	\label{tab:L1T1}
	\centering
	\vspace{0.5cm}
	\begin{tabularx}{6cm}{CCC}
		\toprule
		$V_B$ (\si{V}) & $I_B$(\si{\ampere}) & $R_B$($\Omega$)\\
		\midrule
		0 & & \\
		4 & & \\
		8 & & \\
		12 & & \\
		16 & & \\
		20 & & \\
		24 & & \\
		28 & & \\
		32 & & \\
		\bottomrule
	\end{tabularx}
\end{table}
\begin{table}[H]
	\caption{Valores experimentales de corriente y voltaje en la resistencia}
	\label{tab:L1T2}
	\centering
	\vspace{0.5cm}
    \begin{tabularx}{6cm}{CCC}
		\toprule
		$V_R$ (\si{V}) & $I_R$(\si{\ampere}) & R($\Omega$)\\
		\midrule
		0 & & \\
		4 & & \\
		8 & & \\
		12 & & \\
		16 & & \\
		20 & & \\
		24 & & \\
		28 & & \\
		32 & & \\
		\bottomrule
	\end{tabularx}
\end{table}

\begin{table}[H]
	\caption{Método indirecto de la ley de Ohm aplicado en un circuito con tres resistencias en serie}
	\label{tab:L1T3}
	\centering
	\vspace{0.5cm}
    \begin{tabularx}{14cm}{CCCCCC}
		\toprule
		& $V$ (\si{V}) & $I$(\si{\milli\ampere}) & $R_{teor}$(\si{\ohm}) & $R_{exp}$(\si{\ohm}) & error (\%)\\
		\midrule
		$R_1$ & & & & & \\
		$R_2$ & & & & & \\
		$R_3$ & & & & & \\
		\bottomrule
	\end{tabularx}
\end{table}

\begin{table}[H]
	\caption{Método indirecto de la ley de Ohm aplicado en un circuito con tres resistencias en paralelo}
	\label{tab:L1T4}
	\centering
	\vspace{0.5cm}
    \begin{tabularx}{14cm}{CCCCCC}
		\toprule
		&$V$ (\si{V}) & $I$(\si{\milli\ampere}) & $R_{teor}$(\si{\ohm}) & $R_{exp}$(\si{\ohm}) & error (\%)\\
		\midrule
		$R_1$ & & & & & \\
		$R_2$ & & & & & \\
		$R_3$ & & & & & \\
		\bottomrule
	\end{tabularx}
\end{table}

\section{Análisis de resultados}

\section{Conclusiones}

Si desea citar algo lo puede hacer así \cite{alexander2006fundamentos}



%----------------------------------------------------------------------------------------
%	BIBLIOGRAPHY
%----------------------------------------------------------------------------------------

\bibliographystyle{ieeetr}

\bibliography{referencias}

%----------------------------------------------------------------------------------------


\end{document}