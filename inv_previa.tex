\documentclass[12pt]{article}
\usepackage[utf8]{inputenc}
\usepackage[spanish,activeacute]{babel}
\usepackage[margin=1in]{geometry}
\usepackage{amsmath,amssymb}
\usepackage{multicol}
\usepackage{siunitx}
\usepackage[american,RPvoltages]{circuitikz} %Este se usa para hacer circuitos
\usepackage{float}
\usepackage{hyperref}
\usetikzlibrary{babel}
\usepackage{tabularx}



% *** GRAPHICS RELATED PACKAGES ***
%
\usepackage{graphicx}
\graphicspath{{../pdf/}{../png/}}
\DeclareGraphicsExtensions{.pdf,.jpg,.png}
\usepackage{subfigure}



% Cambio de nombre de cuadro a Tabla
\renewcommand{\listtablename}{Índice de tablas}
\renewcommand{\tablename}{Tabla}

\begin{document}
\noindent
\begin{tabularx}{\linewidth}{Xr}
\textbf{Ingeniería en Mantenimiento Industrial}& \textbf{Investigación previa} \\
\textbf{Escuela de Ingeniería Electromecánica}& \textbf{Laboratorio \# 1}\\
\textbf{Tecnológico de Costa Rica}& \textbf{Lab. de Electricidad I} \\
\end{tabularx}\\

\noindent\rule[2ex]{\textwidth}{2pt}
\begin{tabularx}{\linewidth}{Xr}
\textbf{Integrantes:} & \\
Nombre Apellido & Carné
\end{tabularx}

\noindent\rule[2ex]{\textwidth}{2pt}

\begin{enumerate}
\item ¿Cómo se debe conectar un multímetro para medir la corriente en una carga?\\
Se debe blablabla
\item	¿Cómo se debe conectar un multímetro para medir el voltaje en una carga?
\item	Según las especificaciones de su multímetro y para los distintos rangos de medición, defina cuál es el porcentaje de error en las lecturas de voltaje y corriente.
\item	¿Qué significa la linealidad de un dispositivo eléctrico? ¿Qué consecuencias tiene esta?
\end{enumerate}

\end{document}
